% Created 2016-11-04 Fri 11:40
\documentclass[11pt]{article}
\usepackage[utf8]{inputenc}
\usepackage[T1]{fontenc}
\usepackage{fixltx2e}
\usepackage{graphicx}
\usepackage{longtable}
\usepackage{float}
\usepackage{wrapfig}
\usepackage{rotating}
\usepackage[normalem]{ulem}
\usepackage{amsmath}
\usepackage{textcomp}
\usepackage{marvosym}
\usepackage{wasysym}
\usepackage{amssymb}
\usepackage{hyperref}
\tolerance=1000
\author{Justin Barclay}
\date{\today}
\title{mack\_report}
\hypersetup{
  pdfkeywords={},
  pdfsubject={},
  pdfcreator={Emacs 25.1.1 (Org mode 8.2.10)}}
\begin{document}

\maketitle
\tableofcontents

\section{Services and Configuration *}
\label{sec-1}

\section{Part 1:}
\label{sec-2}
\textbf{Linux root password:}
Qni@rjeIy4NwI6nLHmo67Wy*yDUOuY4\$ynyRmFoJ\%6fy3q\$m\%TwqOTeffo*W40vh
\textbf{Linux student password:}
In\&JRDhntDg5gnTlx7RKSxA8I4!*yey60fOyG4q7uRmRJ@8qi1g4ZHyEd3jQEaXC
\textbf{Linux professor password:}
bzoXuICOJBcGupZjKcySdp74pxO6Npyfv5XTQQv9x3pq7Yr75
\textbf{Windows Student password:}
\#Are there other windows account passwords, and is windows admin the same as Linux?
\textbf{Windows Alice password:}
\$6zv7yMX4!O55j\$z@Yy1h2614o37AaE7r1wY9eH6M102X7184pm!qnzJFB2f06oM
\textbf{Windows Bob password:}
UgG822a2y5f9K7i2sIGlPNa118*99qWqFY0!3$^{\text{GTnPprWTq0dA}}$$^{\text{13e9kH5A0E901}}$
\textbf{NOTE:} please contact us if you use lastpass, and we can share the passwords
with you to make life a little easier (because these are truly sadistic
passwords) without the use of a password manager
\subsection{Reasoning:}
\label{sec-2-1}
\begin{itemize}
\item Passwords are randomly generated using LastPass (very reputable password
service) to prevent dictionary/rainbow attacks and ensure integrity of random
generation
\item We wanted to ensure that it was impossible to brute force a password
\begin{itemize}
\item It takes 16.69 million trillion trillion trillion trillion trillion
\end{itemize}
trillion trillion centuries to exhaustively search this password space
(Assuming one hundred trillion guesses per second)
\item The ease of use, and proven security benefits of using a password manager
and randomly generated passwords makes the use of user created passwords
inexcuseable
\item If we were implimenting this assignment in the real world, these ports
would be open to online brute force/dictionary/rainbow attacks (excepting the use
of something like Fail2Ban), so it is a neccessity to have an extremely
secure password to prevent unauthorized access to the server(s)
\end{itemize}

\section{Part 2}
\label{sec-3}
\subsection{How did you select the passwords for the user accounts on Linux and Windows}
\label{sec-3-1}
\begin{itemize}
\item Passwords are randomly generated using LastPass (very reputable password
service) to prevent dictionary/rainbow attacks and ensure integrity of random
generation
\item We wanted to ensure that it was impossible to brute force a password
\begin{itemize}
\item It takes 16.69 million trillion trillion trillion trillion trillion
\end{itemize}
trillion trillion centuries to exhaustively search this password space
(Assuming one hundred trillion guesses per second)
\item The ease of use, and proven security benefits of using a password manager
and randomly generated passwords makes the use of user created passwords
inexcuseable, as compromising a user account can give an attacker a
foothold to gain higher access privileges on the server
\item If we were implimenting this assignment in the real world, these ports
would be open to online brute force/dictionary/rainbow attacks (excepting the use
of something like Fail2Ban), so it is a neccessity to have an extremely
secure password to prevent unauthorized access to the server(s)
\end{itemize}
\subsection{Compare and contrast how the authentication in the web server works, ftp server, and in the tftp server}
\label{sec-3-2}
\subsubsection{Web Server:}
\label{sec-3-2-1}
\begin{itemize}
\item Authentication in the Apache web server uses mod$_{\text{auth}}$$_{\text{sspi}}$, which is middleware
\end{itemize}
to use Microsoft's NTLM stack for authentication in the Window's encvironment/
\begin{itemize}
\item \textbf{NTLM:} uses a challenege response scheme for authentication
\begin{enumerate}
\item The client negotiates a connection to the server
\item The server responds with a challenge to identify the client
\item The client responds to the challenge with an authentication message, which
is most likely an encrypted or hashed version of a username and password.
\item The server responds indicating a success or failure.
\end{enumerate}
\end{itemize}
\subsubsection{Ftp Server:}
\label{sec-3-2-2}
\begin{itemize}
\item Implimented by VSFTPD, which supports virtual users with PAM (pluggable
authentication modules)
\begin{itemize}
\item \textbf{Linux PAM:} separates the tasks of authentication into four independent
management groups
\begin{enumerate}
\item \uline{Account modules} check that the specified account is a valid
\end{enumerate}
authentication target under current conditions. (may include
conditions like account expiration, time of day, and that the user
has access to the requested service)
\begin{enumerate}
\item \uline{Authentication modules} verify the user's identity, for example
\end{enumerate}
by requesting and checking a, in this case, password.
\begin{enumerate}
\item \uline{Password modules} are responsible for updating passwords, and are
\end{enumerate}
generally coupled to modules in the authentication step. may also be used 
to enforce strong passwords
\begin{enumerate}
\item \uline{Session modules} define actions that are performed at the
\end{enumerate}
beginning and end of sessions. A session starts after a user has
authenticated
\item A virtual user is a user login which does not exist as a real login on the
\end{itemize}
system in /etc/passwd and /etc/shadow file. Virtual users can therefore be
more secure than real users, because a compromised account can only use the
FTP server but cannot login to system to use other services such as SSH or
SMTP)
\end{itemize}
\subsubsection{Tftp Server:}
\label{sec-3-2-3}
\begin{itemize}
\item OpenTFTP server (and TFTP in general)includes no login or access control mechanisms, and thereby provides
no authentication of users. 
\begin{itemize}
\item It is accessible by
\end{itemize}
any anonymous user. It does not provide the ability to manipulate what
directory the user has access to, so upon setting up a tftp connection,
the user only has accesss to the contents of the specified directory
\end{itemize}
\subsubsection{Compare \& Contrast:}
\label{sec-3-2-4}
\begin{itemize}
\item FTP provides the ability to authenticate, and control what users can access
the server contents, whereas the TFTP server only allows control of what
directory a user has access to
\item FTP and TFTP both provide control of what directories a user is able to access
\item FTP and TFTP both do not allow remote code execution in the form of shell access
\item FTP does allow the user to navigate directories within the space they are
permitted to access
\item Web Server relies on using third party middleware, to take advantage of the 
host's built in authentication scheme.
\end{itemize}

\section{Part 3}
\label{sec-4}

\section{Delegation of duties}
\label{sec-5}
\begin{itemize}
\item All group members collaborated on all portions of the assignment, however
each member assumed a leadership role/responsibiltiy for the completion of
one portion of the assignment
\item Justin was in charge of setting up the web server on the Windows machine,
and had a secondary role configuring ip tables, as well as configuring the
Linux machine to support ip tables
\item Vuk was in charge of cracking the passwords for the Assignment 1 sliding
portion, as well as configuring ip tables
\item Mackenzie was in charge of setting up the FTP and TFTP servers on the linux
machine, as well as configuring NAT, and port forwarding for the web server
\end{itemize}

\section{Difficulties Encountered}
\label{sec-6}
\begin{itemize}
\item Testing the iptables to ensure proper function was difficult to manage
\begin{itemize}
\item In the future, it would be good to have a lab day dedicated to testing
\end{itemize}
to make it easy to collaborate with other groups to test eachother's
setups
\item Setting up the Slackware provided TFTP server proved to be impossible, and
the OpenTFTP server had to be installed instead. This also proved somewhat
difficult but not insurmountable
\begin{itemize}
\item In the future, specifically say not to setup the Slackware TFTP server,
\end{itemize}
but others did not have the same trouble that Mackenzie did, so perhaps
he is just not very smart
\item Assignment forced us to learn more about linux permissions, and how
firewalls work
\item The assignment took a reasonable amount of effort, but less so than
Assignment 1
\item The workload was reasonable
\end{itemize}

\section{Resources:}
\label{sec-7}
\url{https://help.ubuntu.com/community/vsftpd}
\url{http://docs.slackware.com/}
\url{http://www.m0rd0r.eu/slackware-as-basic-tftp-server/}
\url{https://en.wikipedia.org/wiki/Linux_PAM}
\url{https://en.wikipedia.org/wiki/NT_LAN_Manager}
% Emacs 25.1.1 (Org mode 8.2.10)
\end{document}
